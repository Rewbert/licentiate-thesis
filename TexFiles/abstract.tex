\begin{raggedright}
\noindent 
{\large\textbf{\mytitle}}\\
\ifx\mysubtitle\undefined\else\textit{\small\mysubtitle}\fi
\end{raggedright}

\vskip 1mm
\noindent
{\sc\authorname}
\vskip 1mm
\noindent
\textit{\mydepartment}\\
\textit{\chalIgu}

\vskip 8mm

\section*{Abstract}

\noindent

In this thesis, we describe our research on how to program low-level platforms with high-level languages.
As an example, consider applications that run on microcontrollers. Such applications
may need to specify precise temporal behavior, carefully manage power usage, and handle cryptographic keys.
Low-level platforms are programmed using low-level languages such as C/C++, where the lack of expressiveness can
lead to error-prone code.

We investigate whether we can use high-level languages to program these platforms, by embedding domain-specific languages
in a host language, Haskell. A high-level language offers better expressivity and shields the developer from low-level
details, yielding code that more concretely describes what the application is supposed to do. Furthermore, a richer runtime
system could ease the burden of e.g. memory management and scheduling of coroutines.

The papers in this thesis indicate that it is possible to program these devices using a high-level language. We develop
two domain-specific languages, Scoria and HasTEE. Scoria is evaluated on NRF52 microcontrollers, where we run applications
that require precise, temporal behavior and perform I/O. HasTEE is evaluated on machines whose processor has support
for Intel Software Guard Extension and shows that the type system of Haskell can be used to automatically partition a Haskell
application and run it in a trusted execution environment.


\vspace{10 mm}
\noindent{\textbf{Keywords}}

\vspace{3 mm}
\noindent{Haskell, Embedded Domain-specific Languages, Confidential Computing, TEE}

