\begin{raggedright}
\noindent 
{\large\textbf{\mytitle}}\\
\ifx\mysubtitle\undefined\else\textit{\small\mysubtitle}\fi
\end{raggedright}

\vskip 1mm
\noindent
{\sc\authorname}
\vskip 1mm
\noindent
\textit{\mydepartment}\\
\textit{\chalIgu}

\vskip 8mm

\section*{Abstract}

\noindent

Without domain-specific language abstractions, developing software for niche domains can be a tedious, error-prone activity.
Notably, when the target platform of your application is a specific piece of hardware, the toolchain for that hardware
very often expects a developer to write low-level C code. C, for all the good it does, does not supply high-level abstractions,
much less domain-specific ones.

In this thesis I present my research on using Haskell to develop domain-specific languages for two specific domains, whose
intended target platform is not traditionally targeted by Haskell developers. The first application domain is real-time applications
for the Internet of Things, where the target platform is a microcontroller. The second application domain is
Confidential Computing, where the target platform is a conventional PC with support for hardware-enforced trusted execution
environments.

\vspace{10 mm}
\noindent{\textbf{Keywords}}

\vspace{3 mm}
\noindent{Haskell, Domain-specific languages, DSL, EDSL, Confidential Computing, Intel SGX, TEE}

