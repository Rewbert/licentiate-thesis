%%=========================================================================%
%%                               Settings                                  %
%%=========================================================================%

%Comment out or remove if it's a Licentiate thesis!
%\def\phdThesis{1}

% Important commands for this thesis!
\newcommand{\currentyear}{2023}
\newcommand{\authorname}{Robert Krook}
\newcommand{\mytitle}{Some nice title}
\newcommand{\mysubtitle}{and an even nicer subtitle} % Comment out if you don't want a subtitle

\newcommand{\division}{Computer Science and Engineering}
%\newcommand{\researchgroup}{Some Research Group} % Comment out if not applicable


%PHD ONLY
\newcommand{\phdISBN}{xxx-xx-xxxx-xxx-x}
\newcommand{\phdSeriesNumber}{xxxx}
\newcommand{\techReportNumber}{XXXX}


%==============%
% Other common strings, these might need to be changed to follow updated guidelines

\newcommand{\licISSN}{1652-876X}
\newcommand{\phdISSN}{0346-718X}

\newcommand{\chalmers}{Chalmers University of Technology}
\newcommand{\gotuni}{University of Gothenburg}
\newcommand{\mydepartment}{Department of Computer Science and Engineering}
\newcommand{\chalIgu}{\chalmers~\IfItalicsTF{$|\!$}{$|$}~\gotuni} % The \! (negative thin space) is to fix kerning because italics seems to add extra space on the right of the inline math
\newcommand{\chalandgu}{\chalmers~and~\gotuni}

\ifx\phdThesis\undefined
\newcommand{\degreetitle}{Licentiate of Engineering}
\else
\newcommand{\degreetitle}{Doctor of Philosophy}
\fi


\ifx\phdThesis\undefined
%reportNo is not used for lic anymore, but double check with the intranet.
%in case you need it just uncomment the text in the identifierNoText command.
\newcommand{\identifierNoText}{%Technical Report No \techReportNumber \\
ISSN \licISSN\\
}
\else
\newcommand{\identifierNoText}{ISBN \phdISBN\\
Doktorsavhandlingar vid Chalmers tekniska h\"{o}gskola, Ny serie nr \phdSeriesNumber .\\
ISSN \phdISSN\\
Technical Report No. \techReportNumber \\
\vspace{1.5cm}
}
\fi




%%=========================================================================%
%%                               Settings                                  %
%%=========================================================================%
\setlength{\headheight}{14mm}

% G5 format
\usepackage{geometry}

\ifx\inlagaPage\undefined
% Thesis is paper size G5
\geometry{paperwidth=169mm, paperheight=239mm, inner=28mm,outer=22mm,top=22mm, bottom=18mm}
\else
% Presentation sheet uses paper size A5
\geometry{paperwidth=148mm, paperheight=210mm, inner=25mm,outer=25mm,top=22mm, bottom=18mm}
\fi

%-----------------------Used packages---------------------------------------------------------%
\usepackage{fancyhdr}
\pagestyle{fancy}
\fancyhead[LE,RO]{\thepage}
\fancyhead[RE]{\scriptsize \slshape \leftmark}
\fancyhead[LO]{\scriptsize \slshape \rightmark}
\fancyfoot[C]{}

\setcounter{secnumdepth}{3} \setcounter{tocdepth}{3}

\usepackage[labelformat=simple]{subcaption}
\renewcommand\thesubfigure{(\alph{subfigure})}
\usepackage{graphicx}
\usepackage{url}
%\usepackage{cite}
% \usepackage{multibib}

%\usepackage[english]{babel}
\usepackage[british]{babel} % Using english will cause dates in the bibliography to be mm-dd-yyyy, instead of the more proper dd-mm-yyyy
\usepackage[backend=bibtex,style=ieee,maxcitenames=3,backref=true,language=british]{biblatex} 
%Different styles depending on whether you want numeric/alphabetic citation key
% style=ieee
% style=apa
\addbibresource{references.bib}

\usepackage{amsmath,amssymb,amscd,latexsym,dsfont}
%\usepackage[latin1]{inputenc}
%\usepackage{amsmath}
%\usepackage{amssymb}
%\usepackage{latexsym}
\usepackage{textcomp}
% \usepackage{multirow}
% \usepackage{multicol}
\usepackage{psfrag}
\usepackage{rotating}
%\usepackage{xspace}
%\usepackage{enumerate}
\usepackage{enumitem}
\usepackage{pdflscape}


\usepackage[final]{pdfpages}
\usepackage{xspace}
\usepackage{relsize}

\usepackage{url}
\usepackage{IEEEtrantools}

% \usepackage{fixltx2e}

\usepackage{microtype}

\usepackage[textsize=tiny,textwidth=20mm,backgroundcolor=orange!70]{todonotes}
\setlength{\marginparwidth}{20mm} % This is so the notes don't get cut out of the page



%----------Some commands you can use-----------%

\newcommand{\figref}[1]{Fig.~\ref{#1}}
\newcommand{\tabref}[1]{Table~\ref{#1}}
\newcommand{\chapref}[1]{Chap.~\ref{#1}}
\newcommand{\secref}[1]{Section~\ref{#1}}
\newcommand{\eqnref}[1]{Equation~\ref{#1}}
\newcommand{\paperref}[1]{Paper~\romannum{#1}}

%use this command to denote figure element that appear inside a figure
\newcommand{\fe}[1]{\textsf{\small #1}}

\newcommand{\p}[1]{\textsf{\small #1}}

\newcommand{\interviewquote}[2]{\begin{quote}
\footnotesize{\emph{``#1'' }} --- \footnotesize{#2}
\end{quote} }



%----------Some commands we need-----------%

\makeatletter
\newcommand*{\IfItalicsTF}{%
  \ifx\f@shape\my@test@it
    \expandafter\@firstoftwo
  \else
    \expandafter\@secondoftwo
  \fi
}
\newcommand*{\my@test@it}{it}
\makeatother


\newcommand{\romannum}[1]{{\uppercase\expandafter{\romannumeral #1\relax}}}

\newcommand{\scalefactorOO}{0.45}

%\linespread{1.0}
% \newcites{ltex}{Appended Publications}
%\addto\captionsenglish{\renewcommand{\figurename}{Fig.}}

% my commands
\newcommand{\code}[1]{\texttt{#1}}